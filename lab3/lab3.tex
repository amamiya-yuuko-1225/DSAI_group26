\documentclass[11pt]{article}

    \usepackage[breakable]{tcolorbox}
    \usepackage{parskip} % Stop auto-indenting (to mimic markdown behaviour)
    

    % Basic figure setup, for now with no caption control since it's done
    % automatically by Pandoc (which extracts ![](path) syntax from Markdown).
    \usepackage{graphicx}
    % Keep aspect ratio if custom image width or height is specified
    \setkeys{Gin}{keepaspectratio}
    % Maintain compatibility with old templates. Remove in nbconvert 6.0
    \let\Oldincludegraphics\includegraphics
    % Ensure that by default, figures have no caption (until we provide a
    % proper Figure object with a Caption API and a way to capture that
    % in the conversion process - todo).
    \usepackage{caption}
    \DeclareCaptionFormat{nocaption}{}
    \captionsetup{format=nocaption,aboveskip=0pt,belowskip=0pt}

    \usepackage{float}
    \floatplacement{figure}{H} % forces figures to be placed at the correct location
    \usepackage{xcolor} % Allow colors to be defined
    \usepackage{enumerate} % Needed for markdown enumerations to work
    \usepackage{geometry} % Used to adjust the document margins
    \usepackage{amsmath} % Equations
    \usepackage{amssymb} % Equations
    \usepackage{textcomp} % defines textquotesingle
    % Hack from http://tex.stackexchange.com/a/47451/13684:
    \AtBeginDocument{%
        \def\PYZsq{\textquotesingle}% Upright quotes in Pygmentized code
    }
    \usepackage{upquote} % Upright quotes for verbatim code
    \usepackage{eurosym} % defines \euro

    \usepackage{iftex}
    \ifPDFTeX
        \usepackage[T1]{fontenc}
        \IfFileExists{alphabeta.sty}{
              \usepackage{alphabeta}
          }{
              \usepackage[mathletters]{ucs}
              \usepackage[utf8x]{inputenc}
          }
    \else
        \usepackage{fontspec}
        \usepackage{unicode-math}
    \fi

    \usepackage{fancyvrb} % verbatim replacement that allows latex
    \usepackage{grffile} % extends the file name processing of package graphics
                         % to support a larger range
    \makeatletter % fix for old versions of grffile with XeLaTeX
    \@ifpackagelater{grffile}{2019/11/01}
    {
      % Do nothing on new versions
    }
    {
      \def\Gread@@xetex#1{%
        \IfFileExists{"\Gin@base".bb}%
        {\Gread@eps{\Gin@base.bb}}%
        {\Gread@@xetex@aux#1}%
      }
    }
    \makeatother
    \usepackage[Export]{adjustbox} % Used to constrain images to a maximum size
    \adjustboxset{max size={0.9\linewidth}{0.9\paperheight}}

    % The hyperref package gives us a pdf with properly built
    % internal navigation ('pdf bookmarks' for the table of contents,
    % internal cross-reference links, web links for URLs, etc.)
    \usepackage{hyperref}
    % The default LaTeX title has an obnoxious amount of whitespace. By default,
    % titling removes some of it. It also provides customization options.
    \usepackage{titling}
    \usepackage{longtable} % longtable support required by pandoc >1.10
    \usepackage{booktabs}  % table support for pandoc > 1.12.2
    \usepackage{array}     % table support for pandoc >= 2.11.3
    \usepackage{calc}      % table minipage width calculation for pandoc >= 2.11.1
    \usepackage[inline]{enumitem} % IRkernel/repr support (it uses the enumerate* environment)
    \usepackage[normalem]{ulem} % ulem is needed to support strikethroughs (\sout)
                                % normalem makes italics be italics, not underlines
    \usepackage{soul}      % strikethrough (\st) support for pandoc >= 3.0.0
    \usepackage{mathrsfs}
    

    
    % Colors for the hyperref package
    \definecolor{urlcolor}{rgb}{0,.145,.698}
    \definecolor{linkcolor}{rgb}{.71,0.21,0.01}
    \definecolor{citecolor}{rgb}{.12,.54,.11}

    % ANSI colors
    \definecolor{ansi-black}{HTML}{3E424D}
    \definecolor{ansi-black-intense}{HTML}{282C36}
    \definecolor{ansi-red}{HTML}{E75C58}
    \definecolor{ansi-red-intense}{HTML}{B22B31}
    \definecolor{ansi-green}{HTML}{00A250}
    \definecolor{ansi-green-intense}{HTML}{007427}
    \definecolor{ansi-yellow}{HTML}{DDB62B}
    \definecolor{ansi-yellow-intense}{HTML}{B27D12}
    \definecolor{ansi-blue}{HTML}{208FFB}
    \definecolor{ansi-blue-intense}{HTML}{0065CA}
    \definecolor{ansi-magenta}{HTML}{D160C4}
    \definecolor{ansi-magenta-intense}{HTML}{A03196}
    \definecolor{ansi-cyan}{HTML}{60C6C8}
    \definecolor{ansi-cyan-intense}{HTML}{258F8F}
    \definecolor{ansi-white}{HTML}{C5C1B4}
    \definecolor{ansi-white-intense}{HTML}{A1A6B2}
    \definecolor{ansi-default-inverse-fg}{HTML}{FFFFFF}
    \definecolor{ansi-default-inverse-bg}{HTML}{000000}

    % common color for the border for error outputs.
    \definecolor{outerrorbackground}{HTML}{FFDFDF}

    % commands and environments needed by pandoc snippets
    % extracted from the output of `pandoc -s`
    \providecommand{\tightlist}{%
      \setlength{\itemsep}{0pt}\setlength{\parskip}{0pt}}
    \DefineVerbatimEnvironment{Highlighting}{Verbatim}{commandchars=\\\{\}}
    % Add ',fontsize=\small' for more characters per line
    \newenvironment{Shaded}{}{}
    \newcommand{\KeywordTok}[1]{\textcolor[rgb]{0.00,0.44,0.13}{\textbf{{#1}}}}
    \newcommand{\DataTypeTok}[1]{\textcolor[rgb]{0.56,0.13,0.00}{{#1}}}
    \newcommand{\DecValTok}[1]{\textcolor[rgb]{0.25,0.63,0.44}{{#1}}}
    \newcommand{\BaseNTok}[1]{\textcolor[rgb]{0.25,0.63,0.44}{{#1}}}
    \newcommand{\FloatTok}[1]{\textcolor[rgb]{0.25,0.63,0.44}{{#1}}}
    \newcommand{\CharTok}[1]{\textcolor[rgb]{0.25,0.44,0.63}{{#1}}}
    \newcommand{\StringTok}[1]{\textcolor[rgb]{0.25,0.44,0.63}{{#1}}}
    \newcommand{\CommentTok}[1]{\textcolor[rgb]{0.38,0.63,0.69}{\textit{{#1}}}}
    \newcommand{\OtherTok}[1]{\textcolor[rgb]{0.00,0.44,0.13}{{#1}}}
    \newcommand{\AlertTok}[1]{\textcolor[rgb]{1.00,0.00,0.00}{\textbf{{#1}}}}
    \newcommand{\FunctionTok}[1]{\textcolor[rgb]{0.02,0.16,0.49}{{#1}}}
    \newcommand{\RegionMarkerTok}[1]{{#1}}
    \newcommand{\ErrorTok}[1]{\textcolor[rgb]{1.00,0.00,0.00}{\textbf{{#1}}}}
    \newcommand{\NormalTok}[1]{{#1}}

    % Additional commands for more recent versions of Pandoc
    \newcommand{\ConstantTok}[1]{\textcolor[rgb]{0.53,0.00,0.00}{{#1}}}
    \newcommand{\SpecialCharTok}[1]{\textcolor[rgb]{0.25,0.44,0.63}{{#1}}}
    \newcommand{\VerbatimStringTok}[1]{\textcolor[rgb]{0.25,0.44,0.63}{{#1}}}
    \newcommand{\SpecialStringTok}[1]{\textcolor[rgb]{0.73,0.40,0.53}{{#1}}}
    \newcommand{\ImportTok}[1]{{#1}}
    \newcommand{\DocumentationTok}[1]{\textcolor[rgb]{0.73,0.13,0.13}{\textit{{#1}}}}
    \newcommand{\AnnotationTok}[1]{\textcolor[rgb]{0.38,0.63,0.69}{\textbf{\textit{{#1}}}}}
    \newcommand{\CommentVarTok}[1]{\textcolor[rgb]{0.38,0.63,0.69}{\textbf{\textit{{#1}}}}}
    \newcommand{\VariableTok}[1]{\textcolor[rgb]{0.10,0.09,0.49}{{#1}}}
    \newcommand{\ControlFlowTok}[1]{\textcolor[rgb]{0.00,0.44,0.13}{\textbf{{#1}}}}
    \newcommand{\OperatorTok}[1]{\textcolor[rgb]{0.40,0.40,0.40}{{#1}}}
    \newcommand{\BuiltInTok}[1]{{#1}}
    \newcommand{\ExtensionTok}[1]{{#1}}
    \newcommand{\PreprocessorTok}[1]{\textcolor[rgb]{0.74,0.48,0.00}{{#1}}}
    \newcommand{\AttributeTok}[1]{\textcolor[rgb]{0.49,0.56,0.16}{{#1}}}
    \newcommand{\InformationTok}[1]{\textcolor[rgb]{0.38,0.63,0.69}{\textbf{\textit{{#1}}}}}
    \newcommand{\WarningTok}[1]{\textcolor[rgb]{0.38,0.63,0.69}{\textbf{\textit{{#1}}}}}


    % Define a nice break command that doesn't care if a line doesn't already
    % exist.
    \def\br{\hspace*{\fill} \\* }
    % Math Jax compatibility definitions
    \def\gt{>}
    \def\lt{<}
    \let\Oldtex\TeX
    \let\Oldlatex\LaTeX
    \renewcommand{\TeX}{\textrm{\Oldtex}}
    \renewcommand{\LaTeX}{\textrm{\Oldlatex}}
    % Document parameters
    % Document title
    \title{lab3}
    
    
    
    
    
    
    
% Pygments definitions
\makeatletter
\def\PY@reset{\let\PY@it=\relax \let\PY@bf=\relax%
    \let\PY@ul=\relax \let\PY@tc=\relax%
    \let\PY@bc=\relax \let\PY@ff=\relax}
\def\PY@tok#1{\csname PY@tok@#1\endcsname}
\def\PY@toks#1+{\ifx\relax#1\empty\else%
    \PY@tok{#1}\expandafter\PY@toks\fi}
\def\PY@do#1{\PY@bc{\PY@tc{\PY@ul{%
    \PY@it{\PY@bf{\PY@ff{#1}}}}}}}
\def\PY#1#2{\PY@reset\PY@toks#1+\relax+\PY@do{#2}}

\@namedef{PY@tok@w}{\def\PY@tc##1{\textcolor[rgb]{0.73,0.73,0.73}{##1}}}
\@namedef{PY@tok@c}{\let\PY@it=\textit\def\PY@tc##1{\textcolor[rgb]{0.24,0.48,0.48}{##1}}}
\@namedef{PY@tok@cp}{\def\PY@tc##1{\textcolor[rgb]{0.61,0.40,0.00}{##1}}}
\@namedef{PY@tok@k}{\let\PY@bf=\textbf\def\PY@tc##1{\textcolor[rgb]{0.00,0.50,0.00}{##1}}}
\@namedef{PY@tok@kp}{\def\PY@tc##1{\textcolor[rgb]{0.00,0.50,0.00}{##1}}}
\@namedef{PY@tok@kt}{\def\PY@tc##1{\textcolor[rgb]{0.69,0.00,0.25}{##1}}}
\@namedef{PY@tok@o}{\def\PY@tc##1{\textcolor[rgb]{0.40,0.40,0.40}{##1}}}
\@namedef{PY@tok@ow}{\let\PY@bf=\textbf\def\PY@tc##1{\textcolor[rgb]{0.67,0.13,1.00}{##1}}}
\@namedef{PY@tok@nb}{\def\PY@tc##1{\textcolor[rgb]{0.00,0.50,0.00}{##1}}}
\@namedef{PY@tok@nf}{\def\PY@tc##1{\textcolor[rgb]{0.00,0.00,1.00}{##1}}}
\@namedef{PY@tok@nc}{\let\PY@bf=\textbf\def\PY@tc##1{\textcolor[rgb]{0.00,0.00,1.00}{##1}}}
\@namedef{PY@tok@nn}{\let\PY@bf=\textbf\def\PY@tc##1{\textcolor[rgb]{0.00,0.00,1.00}{##1}}}
\@namedef{PY@tok@ne}{\let\PY@bf=\textbf\def\PY@tc##1{\textcolor[rgb]{0.80,0.25,0.22}{##1}}}
\@namedef{PY@tok@nv}{\def\PY@tc##1{\textcolor[rgb]{0.10,0.09,0.49}{##1}}}
\@namedef{PY@tok@no}{\def\PY@tc##1{\textcolor[rgb]{0.53,0.00,0.00}{##1}}}
\@namedef{PY@tok@nl}{\def\PY@tc##1{\textcolor[rgb]{0.46,0.46,0.00}{##1}}}
\@namedef{PY@tok@ni}{\let\PY@bf=\textbf\def\PY@tc##1{\textcolor[rgb]{0.44,0.44,0.44}{##1}}}
\@namedef{PY@tok@na}{\def\PY@tc##1{\textcolor[rgb]{0.41,0.47,0.13}{##1}}}
\@namedef{PY@tok@nt}{\let\PY@bf=\textbf\def\PY@tc##1{\textcolor[rgb]{0.00,0.50,0.00}{##1}}}
\@namedef{PY@tok@nd}{\def\PY@tc##1{\textcolor[rgb]{0.67,0.13,1.00}{##1}}}
\@namedef{PY@tok@s}{\def\PY@tc##1{\textcolor[rgb]{0.73,0.13,0.13}{##1}}}
\@namedef{PY@tok@sd}{\let\PY@it=\textit\def\PY@tc##1{\textcolor[rgb]{0.73,0.13,0.13}{##1}}}
\@namedef{PY@tok@si}{\let\PY@bf=\textbf\def\PY@tc##1{\textcolor[rgb]{0.64,0.35,0.47}{##1}}}
\@namedef{PY@tok@se}{\let\PY@bf=\textbf\def\PY@tc##1{\textcolor[rgb]{0.67,0.36,0.12}{##1}}}
\@namedef{PY@tok@sr}{\def\PY@tc##1{\textcolor[rgb]{0.64,0.35,0.47}{##1}}}
\@namedef{PY@tok@ss}{\def\PY@tc##1{\textcolor[rgb]{0.10,0.09,0.49}{##1}}}
\@namedef{PY@tok@sx}{\def\PY@tc##1{\textcolor[rgb]{0.00,0.50,0.00}{##1}}}
\@namedef{PY@tok@m}{\def\PY@tc##1{\textcolor[rgb]{0.40,0.40,0.40}{##1}}}
\@namedef{PY@tok@gh}{\let\PY@bf=\textbf\def\PY@tc##1{\textcolor[rgb]{0.00,0.00,0.50}{##1}}}
\@namedef{PY@tok@gu}{\let\PY@bf=\textbf\def\PY@tc##1{\textcolor[rgb]{0.50,0.00,0.50}{##1}}}
\@namedef{PY@tok@gd}{\def\PY@tc##1{\textcolor[rgb]{0.63,0.00,0.00}{##1}}}
\@namedef{PY@tok@gi}{\def\PY@tc##1{\textcolor[rgb]{0.00,0.52,0.00}{##1}}}
\@namedef{PY@tok@gr}{\def\PY@tc##1{\textcolor[rgb]{0.89,0.00,0.00}{##1}}}
\@namedef{PY@tok@ge}{\let\PY@it=\textit}
\@namedef{PY@tok@gs}{\let\PY@bf=\textbf}
\@namedef{PY@tok@ges}{\let\PY@bf=\textbf\let\PY@it=\textit}
\@namedef{PY@tok@gp}{\let\PY@bf=\textbf\def\PY@tc##1{\textcolor[rgb]{0.00,0.00,0.50}{##1}}}
\@namedef{PY@tok@go}{\def\PY@tc##1{\textcolor[rgb]{0.44,0.44,0.44}{##1}}}
\@namedef{PY@tok@gt}{\def\PY@tc##1{\textcolor[rgb]{0.00,0.27,0.87}{##1}}}
\@namedef{PY@tok@err}{\def\PY@bc##1{{\setlength{\fboxsep}{\string -\fboxrule}\fcolorbox[rgb]{1.00,0.00,0.00}{1,1,1}{\strut ##1}}}}
\@namedef{PY@tok@kc}{\let\PY@bf=\textbf\def\PY@tc##1{\textcolor[rgb]{0.00,0.50,0.00}{##1}}}
\@namedef{PY@tok@kd}{\let\PY@bf=\textbf\def\PY@tc##1{\textcolor[rgb]{0.00,0.50,0.00}{##1}}}
\@namedef{PY@tok@kn}{\let\PY@bf=\textbf\def\PY@tc##1{\textcolor[rgb]{0.00,0.50,0.00}{##1}}}
\@namedef{PY@tok@kr}{\let\PY@bf=\textbf\def\PY@tc##1{\textcolor[rgb]{0.00,0.50,0.00}{##1}}}
\@namedef{PY@tok@bp}{\def\PY@tc##1{\textcolor[rgb]{0.00,0.50,0.00}{##1}}}
\@namedef{PY@tok@fm}{\def\PY@tc##1{\textcolor[rgb]{0.00,0.00,1.00}{##1}}}
\@namedef{PY@tok@vc}{\def\PY@tc##1{\textcolor[rgb]{0.10,0.09,0.49}{##1}}}
\@namedef{PY@tok@vg}{\def\PY@tc##1{\textcolor[rgb]{0.10,0.09,0.49}{##1}}}
\@namedef{PY@tok@vi}{\def\PY@tc##1{\textcolor[rgb]{0.10,0.09,0.49}{##1}}}
\@namedef{PY@tok@vm}{\def\PY@tc##1{\textcolor[rgb]{0.10,0.09,0.49}{##1}}}
\@namedef{PY@tok@sa}{\def\PY@tc##1{\textcolor[rgb]{0.73,0.13,0.13}{##1}}}
\@namedef{PY@tok@sb}{\def\PY@tc##1{\textcolor[rgb]{0.73,0.13,0.13}{##1}}}
\@namedef{PY@tok@sc}{\def\PY@tc##1{\textcolor[rgb]{0.73,0.13,0.13}{##1}}}
\@namedef{PY@tok@dl}{\def\PY@tc##1{\textcolor[rgb]{0.73,0.13,0.13}{##1}}}
\@namedef{PY@tok@s2}{\def\PY@tc##1{\textcolor[rgb]{0.73,0.13,0.13}{##1}}}
\@namedef{PY@tok@sh}{\def\PY@tc##1{\textcolor[rgb]{0.73,0.13,0.13}{##1}}}
\@namedef{PY@tok@s1}{\def\PY@tc##1{\textcolor[rgb]{0.73,0.13,0.13}{##1}}}
\@namedef{PY@tok@mb}{\def\PY@tc##1{\textcolor[rgb]{0.40,0.40,0.40}{##1}}}
\@namedef{PY@tok@mf}{\def\PY@tc##1{\textcolor[rgb]{0.40,0.40,0.40}{##1}}}
\@namedef{PY@tok@mh}{\def\PY@tc##1{\textcolor[rgb]{0.40,0.40,0.40}{##1}}}
\@namedef{PY@tok@mi}{\def\PY@tc##1{\textcolor[rgb]{0.40,0.40,0.40}{##1}}}
\@namedef{PY@tok@il}{\def\PY@tc##1{\textcolor[rgb]{0.40,0.40,0.40}{##1}}}
\@namedef{PY@tok@mo}{\def\PY@tc##1{\textcolor[rgb]{0.40,0.40,0.40}{##1}}}
\@namedef{PY@tok@ch}{\let\PY@it=\textit\def\PY@tc##1{\textcolor[rgb]{0.24,0.48,0.48}{##1}}}
\@namedef{PY@tok@cm}{\let\PY@it=\textit\def\PY@tc##1{\textcolor[rgb]{0.24,0.48,0.48}{##1}}}
\@namedef{PY@tok@cpf}{\let\PY@it=\textit\def\PY@tc##1{\textcolor[rgb]{0.24,0.48,0.48}{##1}}}
\@namedef{PY@tok@c1}{\let\PY@it=\textit\def\PY@tc##1{\textcolor[rgb]{0.24,0.48,0.48}{##1}}}
\@namedef{PY@tok@cs}{\let\PY@it=\textit\def\PY@tc##1{\textcolor[rgb]{0.24,0.48,0.48}{##1}}}

\def\PYZbs{\char`\\}
\def\PYZus{\char`\_}
\def\PYZob{\char`\{}
\def\PYZcb{\char`\}}
\def\PYZca{\char`\^}
\def\PYZam{\char`\&}
\def\PYZlt{\char`\<}
\def\PYZgt{\char`\>}
\def\PYZsh{\char`\#}
\def\PYZpc{\char`\%}
\def\PYZdl{\char`\$}
\def\PYZhy{\char`\-}
\def\PYZsq{\char`\'}
\def\PYZdq{\char`\"}
\def\PYZti{\char`\~}
% for compatibility with earlier versions
\def\PYZat{@}
\def\PYZlb{[}
\def\PYZrb{]}
\makeatother


    % For linebreaks inside Verbatim environment from package fancyvrb.
    \makeatletter
        \newbox\Wrappedcontinuationbox
        \newbox\Wrappedvisiblespacebox
        \newcommand*\Wrappedvisiblespace {\textcolor{red}{\textvisiblespace}}
        \newcommand*\Wrappedcontinuationsymbol {\textcolor{red}{\llap{\tiny$\m@th\hookrightarrow$}}}
        \newcommand*\Wrappedcontinuationindent {3ex }
        \newcommand*\Wrappedafterbreak {\kern\Wrappedcontinuationindent\copy\Wrappedcontinuationbox}
        % Take advantage of the already applied Pygments mark-up to insert
        % potential linebreaks for TeX processing.
        %        {, <, #, %, $, ' and ": go to next line.
        %        _, }, ^, &, >, - and ~: stay at end of broken line.
        % Use of \textquotesingle for straight quote.
        \newcommand*\Wrappedbreaksatspecials {%
            \def\PYGZus{\discretionary{\char`\_}{\Wrappedafterbreak}{\char`\_}}%
            \def\PYGZob{\discretionary{}{\Wrappedafterbreak\char`\{}{\char`\{}}%
            \def\PYGZcb{\discretionary{\char`\}}{\Wrappedafterbreak}{\char`\}}}%
            \def\PYGZca{\discretionary{\char`\^}{\Wrappedafterbreak}{\char`\^}}%
            \def\PYGZam{\discretionary{\char`\&}{\Wrappedafterbreak}{\char`\&}}%
            \def\PYGZlt{\discretionary{}{\Wrappedafterbreak\char`\<}{\char`\<}}%
            \def\PYGZgt{\discretionary{\char`\>}{\Wrappedafterbreak}{\char`\>}}%
            \def\PYGZsh{\discretionary{}{\Wrappedafterbreak\char`\#}{\char`\#}}%
            \def\PYGZpc{\discretionary{}{\Wrappedafterbreak\char`\%}{\char`\%}}%
            \def\PYGZdl{\discretionary{}{\Wrappedafterbreak\char`\$}{\char`\$}}%
            \def\PYGZhy{\discretionary{\char`\-}{\Wrappedafterbreak}{\char`\-}}%
            \def\PYGZsq{\discretionary{}{\Wrappedafterbreak\textquotesingle}{\textquotesingle}}%
            \def\PYGZdq{\discretionary{}{\Wrappedafterbreak\char`\"}{\char`\"}}%
            \def\PYGZti{\discretionary{\char`\~}{\Wrappedafterbreak}{\char`\~}}%
        }
        % Some characters . , ; ? ! / are not pygmentized.
        % This macro makes them "active" and they will insert potential linebreaks
        \newcommand*\Wrappedbreaksatpunct {%
            \lccode`\~`\.\lowercase{\def~}{\discretionary{\hbox{\char`\.}}{\Wrappedafterbreak}{\hbox{\char`\.}}}%
            \lccode`\~`\,\lowercase{\def~}{\discretionary{\hbox{\char`\,}}{\Wrappedafterbreak}{\hbox{\char`\,}}}%
            \lccode`\~`\;\lowercase{\def~}{\discretionary{\hbox{\char`\;}}{\Wrappedafterbreak}{\hbox{\char`\;}}}%
            \lccode`\~`\:\lowercase{\def~}{\discretionary{\hbox{\char`\:}}{\Wrappedafterbreak}{\hbox{\char`\:}}}%
            \lccode`\~`\?\lowercase{\def~}{\discretionary{\hbox{\char`\?}}{\Wrappedafterbreak}{\hbox{\char`\?}}}%
            \lccode`\~`\!\lowercase{\def~}{\discretionary{\hbox{\char`\!}}{\Wrappedafterbreak}{\hbox{\char`\!}}}%
            \lccode`\~`\/\lowercase{\def~}{\discretionary{\hbox{\char`\/}}{\Wrappedafterbreak}{\hbox{\char`\/}}}%
            \catcode`\.\active
            \catcode`\,\active
            \catcode`\;\active
            \catcode`\:\active
            \catcode`\?\active
            \catcode`\!\active
            \catcode`\/\active
            \lccode`\~`\~
        }
    \makeatother

    \let\OriginalVerbatim=\Verbatim
    \makeatletter
    \renewcommand{\Verbatim}[1][1]{%
        %\parskip\z@skip
        \sbox\Wrappedcontinuationbox {\Wrappedcontinuationsymbol}%
        \sbox\Wrappedvisiblespacebox {\FV@SetupFont\Wrappedvisiblespace}%
        \def\FancyVerbFormatLine ##1{\hsize\linewidth
            \vtop{\raggedright\hyphenpenalty\z@\exhyphenpenalty\z@
                \doublehyphendemerits\z@\finalhyphendemerits\z@
                \strut ##1\strut}%
        }%
        % If the linebreak is at a space, the latter will be displayed as visible
        % space at end of first line, and a continuation symbol starts next line.
        % Stretch/shrink are however usually zero for typewriter font.
        \def\FV@Space {%
            \nobreak\hskip\z@ plus\fontdimen3\font minus\fontdimen4\font
            \discretionary{\copy\Wrappedvisiblespacebox}{\Wrappedafterbreak}
            {\kern\fontdimen2\font}%
        }%

        % Allow breaks at special characters using \PYG... macros.
        \Wrappedbreaksatspecials
        % Breaks at punctuation characters . , ; ? ! and / need catcode=\active
        \OriginalVerbatim[#1,codes*=\Wrappedbreaksatpunct]%
    }
    \makeatother

    % Exact colors from NB
    \definecolor{incolor}{HTML}{303F9F}
    \definecolor{outcolor}{HTML}{D84315}
    \definecolor{cellborder}{HTML}{CFCFCF}
    \definecolor{cellbackground}{HTML}{F7F7F7}

    % prompt
    \makeatletter
    \newcommand{\boxspacing}{\kern\kvtcb@left@rule\kern\kvtcb@boxsep}
    \makeatother
    \newcommand{\prompt}[4]{
        {\ttfamily\llap{{\color{#2}[#3]:\hspace{3pt}#4}}\vspace{-\baselineskip}}
    }
    

    
    % Prevent overflowing lines due to hard-to-break entities
    \sloppy
    % Setup hyperref package
    \hypersetup{
      breaklinks=true,  % so long urls are correctly broken across lines
      colorlinks=true,
      urlcolor=urlcolor,
      linkcolor=linkcolor,
      citecolor=citecolor,
      }
    % Slightly bigger margins than the latex defaults
    
    \geometry{verbose,tmargin=1in,bmargin=1in,lmargin=1in,rmargin=1in}
    
    

\begin{document}
    
    \maketitle
    
    

    
    \hypertarget{dat565dit407-assignment-3}{%
\section{DAT565/DIT407 Assignment 3}\label{dat565dit407-assignment-3}}

Author: Group 26 \textbar{} Wenjun Tian wenjunt@chalmers.se \textbar{}
Yifan Tang yifant@chalmers.se

Date: 2024-11-20

    \hypertarget{problem-1-spam-ham}{%
\section{Problem 1: Spam \& Ham}\label{problem-1-spam-ham}}

\hypertarget{a.-data-exploration}{%
\subsection{A. Data exploration}\label{a.-data-exploration}}

By the judgment as a human being, these following features makes me able
to tell spam apart from ham:

\begin{enumerate}
\def\labelenumi{\arabic{enumi}.}
\item
  Topic and related key words: Spam emails mostly focus on commodity
  promotion, dating/porn website promotion, and pure scam. For commodity
  promotion emails, there are key words related to prices, such as
  ``\$'', ``cash'', and ``cheap''; for dating/porn website promotion,
  there are key words concerning sexual features, such as ``amateur'',
  ``wives'', and ``girls''; for scam emails, mostly, might focus on
  topics like unexpected fortune or job opportunities. By contrast, ham
  emails have various topics, including work-related content, personal
  communication, and legitimate newsletters.
\item
  Structure of the \texttt{HTML} content: Spam emails are usually
  generated from a fixed template, thus most of them have complex and
  fancy \texttt{HTML} structures. These can include various fonts,
  colors, and embedded media. On the other hand, ham emails are simple
  and neat in most cases, which focuses on direct communication.
\item
  Spam markers: Spam emails tend to be marked as AD by mail server
  admins and users, while ham emails do not. Moreover, spam emails
  sometimes explicitly assert ``This is NOT spam!'' or something
  similar, which is a very poor lie that reveals the truth. On the other
  hand, ham emails lack such markers and are usually consistent in terms
  of content and senders.
\end{enumerate}

Furthermore, the reasons that make hard ham emails different from easy
ham emails but similar to spam emails are as follows:

\begin{enumerate}
\def\labelenumi{\arabic{enumi}.}
\item
  Similar content to spams: Hard ham emails also focus on promotion of
  commodities, companies, etc. This makes hard ham emails look like
  spams, especially in their subject and promotion words.
\item
  Created from templates: Hard ham emails are also created from
  templates and sent to a large amount of people, which resembles spams
  in terms of \texttt{HTML} structure.
\end{enumerate}

Though hard ham emails are hard to distinguish from spams, there is a
key feature that differentiates those two: Most hard ham emails have
``unsubcribe'' key word, meaning that the receivers can reject further
emailing. However, spam emails rarely provide such function.

\hypertarget{b.-data-splitting}{%
\subsection{B. Data splitting}\label{b.-data-splitting}}

We perform the train-test set split with a ratio of 3:1 by invoking
\texttt{train\_test\_split()} function in
\texttt{sklean.model\_selection} package.

    \hypertarget{problem-2-preprocessing}{%
\section{Problem 2: Preprocessing}\label{problem-2-preprocessing}}

We read email files from specified categories (\texttt{easy\_ham},
\texttt{hard\_ham}, and \texttt{spam}) and stores their content in a
\texttt{DataFrame} for further analysis.

In the \texttt{read\_email\_file()} function, we try to read the content
of an email file using different encodings (ascii, iso-8859-1, and
utf-8) to handle potential encoding issues. If one encoding fails, the
function attempts the next one, ensuring that most email files can be
read without errors.

For each email file, the content is read using
\texttt{read\_email\_file()} and a dictionary containing the content
(\texttt{Content}) and its category (\texttt{Category}) is created.

After reading all email files, we convert the list of dictionaries into
a \texttt{DataFrame} where each line represents an email file.

    \begin{tcolorbox}[breakable, size=fbox, boxrule=1pt, pad at break*=1mm,colback=cellbackground, colframe=cellborder]
\prompt{In}{incolor}{7}{\boxspacing}
\begin{Verbatim}[commandchars=\\\{\}]
\PY{k+kn}{import} \PY{n+nn}{os}
\PY{k+kn}{import} \PY{n+nn}{pandas} \PY{k}{as} \PY{n+nn}{pd}

\PY{l+s+sd}{\PYZsq{}\PYZsq{}\PYZsq{}}
\PY{l+s+sd}{description: read email file, return the content}
\PY{l+s+sd}{param \PYZob{}str\PYZcb{} file\PYZus{}path}
\PY{l+s+sd}{return \PYZob{}str\PYZcb{} content of file }
\PY{l+s+sd}{\PYZsq{}\PYZsq{}\PYZsq{}}
\PY{k}{def} \PY{n+nf}{read\PYZus{}email\PYZus{}file}\PY{p}{(}\PY{n}{file\PYZus{}path}\PY{p}{:} \PY{n+nb}{str}\PY{p}{)} \PY{o}{\PYZhy{}}\PY{o}{\PYZgt{}} \PY{n+nb}{str}\PY{p}{:}
    \PY{k}{try}\PY{p}{:}
        \PY{k}{with} \PY{n+nb}{open}\PY{p}{(}\PY{n}{file\PYZus{}path}\PY{p}{,} \PY{l+s+s1}{\PYZsq{}}\PY{l+s+s1}{r}\PY{l+s+s1}{\PYZsq{}}\PY{p}{,} \PY{n}{encoding}\PY{o}{=}\PY{l+s+s1}{\PYZsq{}}\PY{l+s+s1}{ascii}\PY{l+s+s1}{\PYZsq{}}\PY{p}{)} \PY{k}{as} \PY{n}{f}\PY{p}{:}
            \PY{n}{content} \PY{o}{=} \PY{n}{f}\PY{o}{.}\PY{n}{read}\PY{p}{(}\PY{p}{)}
        \PY{k}{return} \PY{n}{content}
    \PY{k}{except} \PY{n+ne}{UnicodeDecodeError} \PY{k}{as} \PY{n}{e}\PY{p}{:}
        \PY{k}{try}\PY{p}{:}
            \PY{k}{with} \PY{n+nb}{open}\PY{p}{(}\PY{n}{file\PYZus{}path}\PY{p}{,} \PY{l+s+s1}{\PYZsq{}}\PY{l+s+s1}{r}\PY{l+s+s1}{\PYZsq{}}\PY{p}{,} \PY{n}{encoding}\PY{o}{=}\PY{l+s+s1}{\PYZsq{}}\PY{l+s+s1}{iso\PYZhy{}8859\PYZhy{}1}\PY{l+s+s1}{\PYZsq{}}\PY{p}{)} \PY{k}{as} \PY{n}{f}\PY{p}{:}
                \PY{n}{content} \PY{o}{=} \PY{n}{f}\PY{o}{.}\PY{n}{read}\PY{p}{(}\PY{p}{)}
            \PY{k}{return} \PY{n}{content} 
        \PY{k}{except} \PY{n+ne}{UnicodeDecodeError} \PY{k}{as} \PY{n}{e}\PY{p}{:}
            \PY{k}{with} \PY{n+nb}{open}\PY{p}{(}\PY{n}{file\PYZus{}path}\PY{p}{,} \PY{l+s+s1}{\PYZsq{}}\PY{l+s+s1}{r}\PY{l+s+s1}{\PYZsq{}}\PY{p}{,} \PY{n}{encoding}\PY{o}{=}\PY{l+s+s1}{\PYZsq{}}\PY{l+s+s1}{utf\PYZhy{}8}\PY{l+s+s1}{\PYZsq{}}\PY{p}{)} \PY{k}{as} \PY{n}{f}\PY{p}{:}
                \PY{n}{content} \PY{o}{=} \PY{n}{f}\PY{o}{.}\PY{n}{read}\PY{p}{(}\PY{p}{)}
            \PY{k}{return} \PY{n}{content} 
    
\PY{n}{categories} \PY{o}{=} \PY{p}{[}\PY{l+s+s2}{\PYZdq{}}\PY{l+s+s2}{easy\PYZus{}ham}\PY{l+s+s2}{\PYZdq{}}\PY{p}{,} \PY{l+s+s2}{\PYZdq{}}\PY{l+s+s2}{hard\PYZus{}ham}\PY{l+s+s2}{\PYZdq{}}\PY{p}{,} \PY{l+s+s2}{\PYZdq{}}\PY{l+s+s2}{spam}\PY{l+s+s2}{\PYZdq{}}\PY{p}{]}

\PY{n}{rows} \PY{o}{=} \PY{p}{[}\PY{p}{]}

\PY{c+c1}{\PYZsh{}read from email files}
\PY{k}{for} \PY{n}{root}\PY{p}{,} \PY{n}{\PYZus{}}\PY{p}{,} \PY{n}{files} \PY{o+ow}{in} \PY{n}{os}\PY{o}{.}\PY{n}{walk}\PY{p}{(}\PY{l+s+s2}{\PYZdq{}}\PY{l+s+s2}{.}\PY{l+s+s2}{\PYZdq{}}\PY{p}{)}\PY{p}{:}
    \PY{k}{for} \PY{n}{category} \PY{o+ow}{in} \PY{n}{categories}\PY{p}{:}
        \PY{k}{if} \PY{n}{category} \PY{o+ow}{in} \PY{n}{root}\PY{p}{:}
            \PY{k}{for} \PY{n}{f} \PY{o+ow}{in} \PY{n}{files}\PY{p}{:}
                \PY{n}{rows}\PY{o}{.}\PY{n}{append}\PY{p}{(}\PY{p}{\PYZob{}}
                    \PY{l+s+s2}{\PYZdq{}}\PY{l+s+s2}{Content}\PY{l+s+s2}{\PYZdq{}}\PY{p}{:} \PY{n}{read\PYZus{}email\PYZus{}file}\PY{p}{(}\PY{n}{os}\PY{o}{.}\PY{n}{path}\PY{o}{.}\PY{n}{join}\PY{p}{(}\PY{n}{root}\PY{p}{,} \PY{n}{f}\PY{p}{)}\PY{p}{)}\PY{p}{,}
                    \PY{l+s+s2}{\PYZdq{}}\PY{l+s+s2}{Category}\PY{l+s+s2}{\PYZdq{}}\PY{p}{:} \PY{n}{category}
                \PY{p}{\PYZcb{}}\PY{p}{)}

\PY{n}{df} \PY{o}{=} \PY{n}{pd}\PY{o}{.}\PY{n}{DataFrame}\PY{p}{(}\PY{n}{rows}\PY{p}{)}

\PY{n}{df}
\end{Verbatim}
\end{tcolorbox}

            \begin{tcolorbox}[breakable, size=fbox, boxrule=.5pt, pad at break*=1mm, opacityfill=0]
\prompt{Out}{outcolor}{7}{\boxspacing}
\begin{Verbatim}[commandchars=\\\{\}]
                                                Content  Category
0     Return-Path: <bounce-lgmedia-2534370@sprocket{\ldots}  hard\_ham
1     Return-Path: <Online\#3.19725.55-A8YAgb1NX5rYkd{\ldots}  hard\_ham
2     Return-Path: <Online\#3.19592.a8-JNyKlW9O8FdiLs{\ldots}  hard\_ham
3     From bounce-neatnettricks-2424157@silver.lyris{\ldots}  hard\_ham
4     Return-Path: <Online\#3.20115.09-rB-TgEkNwY9w6R{\ldots}  hard\_ham
{\ldots}                                                 {\ldots}       {\ldots}
3297  From Alex-09242002-HTML@frugaljoe.330w.com  Th{\ldots}      spam
3298  From mando@insiq.us  Mon Aug 26 15:49:52 2002\textbackslash{}{\ldots}      spam
3299  Return-Path: ler@lerami.lerctr.org\textbackslash{}nDelivery-D{\ldots}      spam
3300  From fholland@bigfoot.com  Wed Sep 11 19:43:52{\ldots}      spam
3301  From arnoldm@aol.com  Mon Sep  9 19:31:32 2002{\ldots}      spam

[3302 rows x 2 columns]
\end{Verbatim}
\end{tcolorbox}
        
    \hypertarget{problem-3-easy-ham}{%
\section{Problem 3: Easy Ham}\label{problem-3-easy-ham}}

\hypertarget{code-logic}{%
\subsection{3.1 Code logic}\label{code-logic}}

First, we perform the train-test set split with a ratio of 3:1 by
invoking \texttt{train\_test\_split()} function.

Then, we use function \texttt{analyze()} to train and evaluate models
for classifying spam versus ham emails.

In this fucntion, it uses \texttt{CountVectorizer} to convert email
content into bag of words vectors(in the form of \texttt{CRS}) and
\texttt{LabelEncoder} to encode the categories into integers
(\texttt{easy\_ham}, \texttt{hard\_ham}, \texttt{spam}).

Note that we apply \texttt{fit\_transform()} on the training set and
only \texttt{transform()} on the test set. The \texttt{fit()} function
is to ``learn'' from the data (e.g., \texttt{CountVectorizer.fit()} is
to create the dictionary of words). Thus it is only applied on the
training set; the \texttt{transform()} function is to transform data
according to the principles learned from \texttt{fit()} (e.g,
\texttt{CountVectorizer.tranform()} is to transform plain text into bag
of words vectors in the form of CRS based on the dictionary learned from
\texttt{fit()}). Thus it is applied on both training and test set.

Then, the function defines \texttt{train\_and\_test} that fits a Naive
Bayes classifier (BernoulliNB or MultinomialNB) to the training data and
evaluates it on the test set. We \texttt{fit()} the training set into
the classifier, and then use the classifier to \texttt{predict()} the
categories of content in the test set.

After that we compare and contrast the predicted and actual categories
of the test set, computing \texttt{accuracy}, \texttt{recall},
\texttt{precision} and the confusion matrix.

    \begin{tcolorbox}[breakable, size=fbox, boxrule=1pt, pad at break*=1mm,colback=cellbackground, colframe=cellborder]
\prompt{In}{incolor}{8}{\boxspacing}
\begin{Verbatim}[commandchars=\\\{\}]
\PY{l+s+sd}{\PYZsq{}\PYZsq{}\PYZsq{}}
\PY{l+s+sd}{Author: amamiya\PYZhy{}yuuko\PYZhy{}1225 1913250675@qq.com}
\PY{l+s+sd}{Date: 2024\PYZhy{}11\PYZhy{}19 17:18:58}
\PY{l+s+sd}{LastEditors: amamiya\PYZhy{}yuuko\PYZhy{}1225 1913250675@qq.com}
\PY{l+s+sd}{Description: }
\PY{l+s+sd}{\PYZsq{}\PYZsq{}\PYZsq{}}
\PY{k+kn}{from} \PY{n+nn}{sklearn}\PY{n+nn}{.}\PY{n+nn}{model\PYZus{}selection} \PY{k+kn}{import} \PY{n}{train\PYZus{}test\PYZus{}split}
\PY{k+kn}{from} \PY{n+nn}{sklearn}\PY{n+nn}{.}\PY{n+nn}{feature\PYZus{}extraction}\PY{n+nn}{.}\PY{n+nn}{text} \PY{k+kn}{import} \PY{n}{CountVectorizer}
\PY{k+kn}{from} \PY{n+nn}{sklearn}\PY{n+nn}{.}\PY{n+nn}{preprocessing} \PY{k+kn}{import} \PY{n}{LabelEncoder}
\PY{k+kn}{from} \PY{n+nn}{sklearn}\PY{n+nn}{.}\PY{n+nn}{naive\PYZus{}bayes} \PY{k+kn}{import} \PY{n}{BernoulliNB}\PY{p}{,} \PY{n}{MultinomialNB}

\PY{l+s+sd}{\PYZsq{}\PYZsq{}\PYZsq{}}
\PY{l+s+sd}{description: analyze based on \PYZdq{}easy\PYZus{}ham\PYZdq{} for q3 or \PYZdq{}hard\PYZus{}ham\PYZdq{} for q4}
\PY{l+s+sd}{param \PYZob{}*\PYZcb{} df\PYZus{}train: training set}
\PY{l+s+sd}{param \PYZob{}*\PYZcb{} df\PYZus{}test test set}
\PY{l+s+sd}{param \PYZob{}*\PYZcb{} type: \PYZdq{}easy\PYZus{}ham\PYZdq{} for q3 or \PYZdq{}hard\PYZus{}ham\PYZdq{} for q4}
\PY{l+s+sd}{return \PYZob{}*\PYZcb{} no return, but print acc, precision, recall, and confusion matrix}
\PY{l+s+sd}{\PYZsq{}\PYZsq{}\PYZsq{}}
\PY{k}{def} \PY{n+nf}{analyze}\PY{p}{(}\PY{n}{df\PYZus{}train}\PY{p}{,} \PY{n}{df\PYZus{}test}\PY{p}{,} \PY{n+nb}{type}\PY{p}{)}\PY{p}{:}
    
    \PY{n}{cv} \PY{o}{=} \PY{n}{CountVectorizer}\PY{p}{(}\PY{p}{)}\PY{p}{;} \PY{n}{le} \PY{o}{=} \PY{n}{LabelEncoder}\PY{p}{(}\PY{p}{)}

    \PY{n}{X\PYZus{}train}\PY{p}{,} \PY{n}{X\PYZus{}test} \PY{o}{=} \PY{n}{cv}\PY{o}{.}\PY{n}{fit\PYZus{}transform}\PY{p}{(}\PY{n}{df\PYZus{}train}\PY{p}{[}\PY{l+s+s1}{\PYZsq{}}\PY{l+s+s1}{Content}\PY{l+s+s1}{\PYZsq{}}\PY{p}{]}\PY{p}{)}\PY{p}{,} \PY{n}{cv}\PY{o}{.}\PY{n}{transform}\PY{p}{(}\PY{n}{df\PYZus{}test}\PY{p}{[}\PY{l+s+s1}{\PYZsq{}}\PY{l+s+s1}{Content}\PY{l+s+s1}{\PYZsq{}}\PY{p}{]}\PY{p}{)}
    \PY{n}{y\PYZus{}train}\PY{p}{,} \PY{n}{y\PYZus{}test} \PY{o}{=} \PY{n}{le}\PY{o}{.}\PY{n}{fit\PYZus{}transform}\PY{p}{(}\PY{n}{df\PYZus{}train}\PY{p}{[}\PY{l+s+s1}{\PYZsq{}}\PY{l+s+s1}{Category}\PY{l+s+s1}{\PYZsq{}}\PY{p}{]}\PY{p}{)}\PY{p}{,} \PY{n}{le}\PY{o}{.}\PY{n}{transform}\PY{p}{(}\PY{n}{df\PYZus{}test}\PY{p}{[}\PY{l+s+s1}{\PYZsq{}}\PY{l+s+s1}{Category}\PY{l+s+s1}{\PYZsq{}}\PY{p}{]}\PY{p}{)}

\PY{+w}{    }\PY{l+s+sd}{\PYZsq{}\PYZsq{}\PYZsq{}}
\PY{l+s+sd}{    description: train and test using specified classifer}
\PY{l+s+sd}{    param \PYZob{}*\PYZcb{} classifier: BernoulliNB or MultinomialNB }
\PY{l+s+sd}{    param \PYZob{}*\PYZcb{} name: name of the classifier}
\PY{l+s+sd}{    return \PYZob{}*\PYZcb{} no return, but print acc, precision, recall, and confusion matrix}
\PY{l+s+sd}{    \PYZsq{}\PYZsq{}\PYZsq{}}
    \PY{k}{def} \PY{n+nf}{train\PYZus{}and\PYZus{}test}\PY{p}{(}\PY{n}{classifier}\PY{p}{,} \PY{n}{name}\PY{p}{)}\PY{p}{:}
        \PY{n}{classifier}\PY{o}{.}\PY{n}{fit}\PY{p}{(}\PY{n}{X\PYZus{}train}\PY{p}{,}\PY{n}{y\PYZus{}train}\PY{p}{)}
        \PY{n}{y\PYZus{}pred} \PY{o}{=} \PY{n}{classifier}\PY{o}{.}\PY{n}{predict}\PY{p}{(}\PY{n}{X\PYZus{}test}\PY{p}{)}
        \PY{n}{y\PYZus{}test\PYZus{}inv} \PY{o}{=} \PY{n}{le}\PY{o}{.}\PY{n}{inverse\PYZus{}transform}\PY{p}{(}\PY{n}{y\PYZus{}test}\PY{p}{)}
        \PY{n}{y\PYZus{}pred\PYZus{}inv} \PY{o}{=} \PY{n}{le}\PY{o}{.}\PY{n}{inverse\PYZus{}transform}\PY{p}{(}\PY{n}{y\PYZus{}pred}\PY{p}{)}
        \PY{n}{tp} \PY{o}{=} \PY{p}{(}\PY{p}{(}\PY{n}{y\PYZus{}test\PYZus{}inv} \PY{o}{==} \PY{l+s+s1}{\PYZsq{}}\PY{l+s+s1}{spam}\PY{l+s+s1}{\PYZsq{}}\PY{p}{)} \PY{o}{\PYZam{}} \PY{p}{(}\PY{n}{y\PYZus{}pred\PYZus{}inv} \PY{o}{==} \PY{l+s+s1}{\PYZsq{}}\PY{l+s+s1}{spam}\PY{l+s+s1}{\PYZsq{}}\PY{p}{)}\PY{p}{)}\PY{o}{.}\PY{n}{sum}\PY{p}{(}\PY{p}{)}
        \PY{n}{fp} \PY{o}{=} \PY{p}{(}\PY{p}{(}\PY{n}{y\PYZus{}test\PYZus{}inv} \PY{o}{==} \PY{n+nb}{type}\PY{p}{)} \PY{o}{\PYZam{}} \PY{p}{(}\PY{n}{y\PYZus{}pred\PYZus{}inv} \PY{o}{==} \PY{l+s+s1}{\PYZsq{}}\PY{l+s+s1}{spam}\PY{l+s+s1}{\PYZsq{}}\PY{p}{)}\PY{p}{)}\PY{o}{.}\PY{n}{sum}\PY{p}{(}\PY{p}{)}
        \PY{n}{fn} \PY{o}{=} \PY{p}{(}\PY{p}{(}\PY{n}{y\PYZus{}test\PYZus{}inv} \PY{o}{==} \PY{l+s+s1}{\PYZsq{}}\PY{l+s+s1}{spam}\PY{l+s+s1}{\PYZsq{}}\PY{p}{)} \PY{o}{\PYZam{}} \PY{p}{(}\PY{n}{y\PYZus{}pred\PYZus{}inv} \PY{o}{==} \PY{n+nb}{type}\PY{p}{)}\PY{p}{)}\PY{o}{.}\PY{n}{sum}\PY{p}{(}\PY{p}{)}
        \PY{n}{tn} \PY{o}{=} \PY{p}{(}\PY{p}{(}\PY{n}{y\PYZus{}test\PYZus{}inv} \PY{o}{==} \PY{n+nb}{type}\PY{p}{)} \PY{o}{\PYZam{}} \PY{p}{(}\PY{n}{y\PYZus{}pred\PYZus{}inv} \PY{o}{==} \PY{n+nb}{type}\PY{p}{)}\PY{p}{)}\PY{o}{.}\PY{n}{sum}\PY{p}{(}\PY{p}{)}
        \PY{n}{acc} \PY{o}{=} \PY{p}{(}\PY{n}{tp}\PY{o}{+}\PY{n}{tn}\PY{p}{)}\PY{o}{/}\PY{p}{(}\PY{n}{tp}\PY{o}{+}\PY{n}{fp}\PY{o}{+}\PY{n}{tn}\PY{o}{+}\PY{n}{fn}\PY{p}{)}
        \PY{n}{precision} \PY{o}{=} \PY{n}{tp}\PY{o}{/}\PY{p}{(}\PY{n}{tp}\PY{o}{+}\PY{n}{fp}\PY{p}{)}
        \PY{n}{recall} \PY{o}{=} \PY{n}{tp}\PY{o}{/}\PY{p}{(}\PY{n}{tp}\PY{o}{+}\PY{n}{fn}\PY{p}{)} 
        \PY{n+nb}{print}\PY{p}{(}\PY{l+s+sa}{f}\PY{l+s+s2}{\PYZdq{}}\PY{l+s+si}{\PYZob{}}\PY{n}{name}\PY{l+s+si}{\PYZcb{}}\PY{l+s+s2}{:}\PY{l+s+se}{\PYZbs{}n}\PY{l+s+s2}{accuracy:}\PY{l+s+si}{\PYZob{}}\PY{n}{acc}\PY{l+s+si}{\PYZcb{}}\PY{l+s+s2}{,precision:}\PY{l+s+si}{\PYZob{}}\PY{n}{precision}\PY{l+s+si}{\PYZcb{}}\PY{l+s+s2}{,recall:}\PY{l+s+si}{\PYZob{}}\PY{n}{recall}\PY{l+s+si}{\PYZcb{}}\PY{l+s+s2}{\PYZdq{}}\PY{p}{)}
        \PY{n+nb}{print}\PY{p}{(}\PY{l+s+sa}{f}\PY{l+s+s2}{\PYZdq{}}\PY{l+s+s2}{confusion matrix:}\PY{l+s+se}{\PYZbs{}n}\PY{l+s+s2}{ tp: }\PY{l+s+si}{\PYZob{}}\PY{n}{tp}\PY{l+s+si}{\PYZcb{}}\PY{l+s+s2}{, fn: }\PY{l+s+si}{\PYZob{}}\PY{n}{fn}\PY{l+s+si}{\PYZcb{}}\PY{l+s+s2}{ }\PY{l+s+se}{\PYZbs{}n}\PY{l+s+s2}{ fp: }\PY{l+s+si}{\PYZob{}}\PY{n}{fp}\PY{l+s+si}{\PYZcb{}}\PY{l+s+s2}{, tn: }\PY{l+s+si}{\PYZob{}}\PY{n}{tn}\PY{l+s+si}{\PYZcb{}}\PY{l+s+s2}{ }\PY{l+s+se}{\PYZbs{}n}\PY{l+s+s2}{\PYZdq{}}\PY{p}{)}

    \PY{n}{train\PYZus{}and\PYZus{}test}\PY{p}{(}\PY{n}{BernoulliNB}\PY{p}{(}\PY{p}{)}\PY{p}{,} \PY{l+s+s2}{\PYZdq{}}\PY{l+s+s2}{BernoulliNB}\PY{l+s+s2}{\PYZdq{}}\PY{p}{)}

    \PY{n}{train\PYZus{}and\PYZus{}test}\PY{p}{(}\PY{n}{MultinomialNB}\PY{p}{(}\PY{p}{)}\PY{p}{,} \PY{l+s+s2}{\PYZdq{}}\PY{l+s+s2}{MultinomialNB}\PY{l+s+s2}{\PYZdq{}}\PY{p}{)}

\PY{c+c1}{\PYZsh{} random seed for training and test set split}
\PY{n}{SEED} \PY{o}{=} \PY{l+m+mi}{1919810}

\PY{c+c1}{\PYZsh{} split training and test set}
\PY{n}{df\PYZus{}train}\PY{p}{,} \PY{n}{df\PYZus{}test} \PY{o}{=} \PY{n}{train\PYZus{}test\PYZus{}split}\PY{p}{(}\PY{n}{df}\PY{p}{[}\PY{n}{df}\PY{p}{[}\PY{l+s+s1}{\PYZsq{}}\PY{l+s+s1}{Category}\PY{l+s+s1}{\PYZsq{}}\PY{p}{]}\PY{o}{.}\PY{n}{isin}\PY{p}{(}\PY{p}{[}\PY{l+s+s1}{\PYZsq{}}\PY{l+s+s1}{easy\PYZus{}ham}\PY{l+s+s1}{\PYZsq{}}\PY{p}{,} \PY{l+s+s1}{\PYZsq{}}\PY{l+s+s1}{spam}\PY{l+s+s1}{\PYZsq{}}\PY{p}{]}\PY{p}{)}\PY{p}{]}\PY{p}{,} \PY{n}{random\PYZus{}state}\PY{o}{=}\PY{n}{SEED}\PY{p}{)}

\PY{n}{analyze}\PY{p}{(}\PY{n}{df\PYZus{}train}\PY{p}{,} \PY{n}{df\PYZus{}test}\PY{p}{,} \PY{l+s+s1}{\PYZsq{}}\PY{l+s+s1}{easy\PYZus{}ham}\PY{l+s+s1}{\PYZsq{}}\PY{p}{)}
\end{Verbatim}
\end{tcolorbox}

    \begin{Verbatim}[commandchars=\\\{\}]
BernoulliNB:
accuracy:0.9043250327653998,precision:0.9166666666666666,recall:0.44715447154471
544
confusion matrix:
 tp: 55, fn: 68
 fp: 5, tn: 635

MultinomialNB:
accuracy:0.9672346002621232,precision:0.9803921568627451,recall:0.81300813008130
08
confusion matrix:
 tp: 100, fn: 23
 fp: 2, tn: 638

    \end{Verbatim}

    \hypertarget{results}{%
\subsection{3.2 Results}\label{results}}

Thus, the results are as follows:

\hypertarget{bernoullinb}{%
\subsubsection{BernoulliNB}\label{bernoullinb}}

accuracy: 0.9043250327653998; precision: 0.9166666666666666; recall:
0.44715447154471544

confusion matrix:

\begin{longtable}[]{@{}llll@{}}
\toprule\noalign{}
& Pred. pos. & Pred. neg. & Marginal sum \\
\midrule\noalign{}
\endhead
\bottomrule\noalign{}
\endlastfoot
Actual pos. & 55 & 68 & 123 \\
Actual neg. & 5 & 635 & 640 \\
Marginal sum & 60 & 703 & \\
\end{longtable}

\hypertarget{multinomialnb}{%
\subsubsection{MultinomialNB}\label{multinomialnb}}

accuracy: 0.9672346002621232; precision: 0.9803921568627451; recall:
0.8130081300813008

confusion matrix:

\begin{longtable}[]{@{}llll@{}}
\toprule\noalign{}
& Pred. pos. & Pred. neg. & Marginal sum \\
\midrule\noalign{}
\endhead
\bottomrule\noalign{}
\endlastfoot
Actual pos. & 100 & 23 & 123 \\
Actual neg. & 2 & 638 & 640 \\
Marginal sum & 102 & 661 & \\
\end{longtable}

    \hypertarget{problem-4-hard-ham}{%
\section{Problem 4: Hard Ham}\label{problem-4-hard-ham}}

\hypertarget{code-logic}{%
\subsection{4.1 Code logic}\label{code-logic}}

The logic is the same as problem 3, just consider \texttt{hard\_ham}
instead of \texttt{easy\_ham}

    \begin{tcolorbox}[breakable, size=fbox, boxrule=1pt, pad at break*=1mm,colback=cellbackground, colframe=cellborder]
\prompt{In}{incolor}{9}{\boxspacing}
\begin{Verbatim}[commandchars=\\\{\}]
\PY{n}{df\PYZus{}train}\PY{p}{,} \PY{n}{df\PYZus{}test} \PY{o}{=} \PY{n}{train\PYZus{}test\PYZus{}split}\PY{p}{(}\PY{n}{df}\PY{p}{[}\PY{n}{df}\PY{p}{[}\PY{l+s+s1}{\PYZsq{}}\PY{l+s+s1}{Category}\PY{l+s+s1}{\PYZsq{}}\PY{p}{]}\PY{o}{.}\PY{n}{isin}\PY{p}{(}\PY{p}{[}\PY{l+s+s1}{\PYZsq{}}\PY{l+s+s1}{hard\PYZus{}ham}\PY{l+s+s1}{\PYZsq{}}\PY{p}{,} \PY{l+s+s1}{\PYZsq{}}\PY{l+s+s1}{spam}\PY{l+s+s1}{\PYZsq{}}\PY{p}{]}\PY{p}{)}\PY{p}{]}\PY{p}{,} \PY{n}{random\PYZus{}state}\PY{o}{=}\PY{n}{SEED}\PY{p}{)}

\PY{n}{analyze}\PY{p}{(}\PY{n}{df\PYZus{}train}\PY{p}{,} \PY{n}{df\PYZus{}test}\PY{p}{,} \PY{l+s+s1}{\PYZsq{}}\PY{l+s+s1}{hard\PYZus{}ham}\PY{l+s+s1}{\PYZsq{}}\PY{p}{)}
\end{Verbatim}
\end{tcolorbox}

    \begin{Verbatim}[commandchars=\\\{\}]
BernoulliNB:
accuracy:0.8882978723404256,precision:0.8642857142857143,recall:0.98373983739837
4
confusion matrix:
 tp: 121, fn: 2
 fp: 19, tn: 46

MultinomialNB:
accuracy:0.9361702127659575,precision:0.9172932330827067,recall:0.99186991869918
7
confusion matrix:
 tp: 122, fn: 1
 fp: 11, tn: 54

    \end{Verbatim}

    \hypertarget{results}{%
\subsection{4.2 Results}\label{results}}

Thus, the results are as follows:

\hypertarget{bernoullinb}{%
\subsubsection{BernoulliNB}\label{bernoullinb}}

accuracy: 0.8882978723404256 ; precision: 0.8642857142857143 ; recall:
0.983739837398374

confusion matrix:

\begin{longtable}[]{@{}llll@{}}
\toprule\noalign{}
& Pred. pos. & Pred. neg. & Marginal sum \\
\midrule\noalign{}
\endhead
\bottomrule\noalign{}
\endlastfoot
Actual pos. & 121 & 2 & 123 \\
Actual neg. & 19 & 46 & 65 \\
Marginal sum & 140 & 48 & \\
\end{longtable}

\hypertarget{multinomialnb}{%
\subsubsection{MultinomialNB}\label{multinomialnb}}

accuracy: 0.9361702127659575 ; precision: 0.9172932330827067 ; recall:
0.991869918699187

confusion matrix:

\begin{longtable}[]{@{}llll@{}}
\toprule\noalign{}
& Pred. pos. & Pred. neg. & Marginal sum \\
\midrule\noalign{}
\endhead
\bottomrule\noalign{}
\endlastfoot
Actual pos. & 122 & 1 & 123 \\
Actual neg. & 11 & 54 & 65 \\
Marginal sum & 133 & 55 & \\
\end{longtable}

\hypertarget{differences-between-problem-3-and-4}{%
\subsection{4.3 Differences between problem 3 and
4}\label{differences-between-problem-3-and-4}}

\begin{enumerate}
\def\labelenumi{\arabic{enumi}.}
\item
  Accuracy: The accuracy for both classifiers for \texttt{easy\_ham} is
  higher than for \texttt{hard\_ham}. It is reasonable since
  \texttt{easy\_ham} is easier to differentiate from spam, while
  \texttt{hard\_ham} often contains promotional content.
\item
  Precision: For \texttt{easy\_ham}, the precision for both classifiers
  is quite high. It implys that most emails predictions of spam are
  correct, while \texttt{hard\_ham} has lower precision according to
  confusing content.
\item
  Recall: The recall is much lower for BernoulliNB on
  \texttt{easy\_ham}, which means it cannot distinguish the spam email
  in many times. On the other hand, \texttt{hard\_ham} shows a higher
  recall for both classifiers. Overall, the recall classifier for
  \texttt{easy\_ham} is lower than for \texttt{hard\_ham}.
\item
  Confusion Matrix: Both classifiers have a higher value of false
  positives (FP) for \texttt{hard\_ham} than for \texttt{easy\_ham},
  indicating that legitimate but promotion emails are often detected as
  spam by mistake. However, both classifiers have a very low false
  negative (FN) value, meaning that actual spam emails are almost always
  correctly identified.
\end{enumerate}


    % Add a bibliography block to the postdoc
    
    
    
\end{document}
